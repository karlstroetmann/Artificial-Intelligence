\chapter{Linear Regression}
\section{Simple Linear Regression}
Assume we have a \emph{linear hypothesis} 
\\[0.2cm]
\hspace*{1.3cm}
$h_\mathbf{\theta}(x) := \theta_1 \cdot x + \theta_0$
\\[0.2cm]
We have $m$ observations $\langle x^{(1)}, y^{(1)}\rangle, \cdots, \langle x^{(m)}, y^{(m)}\rangle$.  Our goal is to minimize the \emph{squared error}
\\[0.2cm]
\hspace*{1.3cm}
$E(\mathbf{\theta}) = \sum\limits_{i=1}^m h_\mathbf{\theta}\bigl(x^{(i)} - y^{(i)}\bigr)^2$
\\[0.2cm]


\section{General Linear Regression}
In a \emph{general regression problem} we are given a list of $n$ pairs of the form $\langle\mathbf{x}^{(i)}, y^{(i)} \rangle$ 
where $\mathbf{x}^{(i)} \in \mathbb{R}^m$ and $y^{(i)} \in \mathbb{R}$ for all $i \in \{1,\cdots,n\}$.  These pairs are called the \emph{training examples}.
Our goal is to compute a linear function 
\\[0.2cm]
\hspace*{1.3cm}
$F:\mathbb{R}^m \rightarrow \mathbb{R}$
\\[0.2cm]  
such that $F\bigl(\mathbf{x}^{(i)}\bigr)$ approximates  $y^{(i)}$ as much as posssible
for all $i\in\{1,\cdots,n\}$, i.e.~we want to have
\\[0.2cm]
\hspace*{1.3cm}
$\forall i\in\{1,\cdots,n\}:F\bigl(\mathbf{x}^{(i)}\bigr) \approx y^{(i)}$.
\\[0.2cm]
In order to make the notation $F\bigl(\mathbf{x}^{(i)}\bigr) \approx y^{(i)}$ more precise, we
define the \emph{squared error} 
\begin{equation}
  \label{eq:squared-error-1}
  E := \frac{1}{n} \cdot \sum\limits_{i=1}^n \Bigl(F\bigl(\mathbf{x}^{(i)}\bigr) - y^{(i)}\Bigr)^2. 
\end{equation}
Then, given the list of training examples $[\langle \mathbf{x}^{(1)}, y^{(1)} \rangle, \cdots, \langle
\mathbf{x}^{n}, y^{(n)} \rangle]$, our goal is to minimize the squared error $E$.  
In order to proceed, we need to have a model for the function $F$.  The simplest model is a linear
model, i.e.~we assume that $F$ is given as 
\\[0.2cm]
\hspace*{1.3cm}
$\ds F(\mathbf{x}) = \sum\limits_{i=1}^m w_i \cdot x_i + b = \mathbf{x}^T \cdot \mathbf{w} + b$ \quad where $\mathbf{w} \in \mathbb{R}^m$ and $b\in\mathbb{R}$.
\\[0.2cm]
Here, the expression $\mathbf{x}^T \cdot \mathbf{w}$ denotes the matrix product of the vector
$\mathbf{x}^T$, which is viewed as a $1$-by-$m$ matrix, and the vector $\mathbf{w}$.
At this point you might wonder why it is useful to introduce matrix notation here.  The reason is
that this notation shortens the formula and, furthermore, is more efficient to implement since most
programming languages used in machine learning have special library support for matrix operations.

The definition of $F$ given above is the model used in
\href{https://en.wikipedia.org/wiki/Linear_regression}{linear regression}. 
Here, $\mathbf{w}$ is called the \emph{weight vector} and $b$ is called the \emph{bias}.  It turns
out that the notation can be simplified if we extend the $m$-dimensional vector $\mathbf{x}$ to an
$m+1$-dimensional vector $\mathbf{x}'$ such that
\\[0.2cm]
\hspace*{1.3cm}
$x_j' := x_j$ \quad for all $j\in\{1,\cdots,m\}$ \quad and \quad $x_{m+1}' := 1$.
\\[0.2cm]
To put it in words, the vector $\mathbf{x}'$ results from the vector $\mathbf{x}$ by appending the number $1$:
\\[0.2cm]
\hspace*{1.3cm}
$\mathbf{x}' = \langle x_1, \cdots, x_m, 1 \rangle^T$ \quad where $\langle x_1, \cdots, x_m \rangle = \mathbf{x}^T$.
\\[0.2cm]
Furthermore, we define 
\\[0.2cm]
\hspace*{1.3cm}
$\mathbf{w}' := \langle w_1, \cdots, w_m, b \rangle^T$ \quad where $\langle w_1, \cdots, w_m \rangle = \mathbf{w}^T$.
\\[0.2cm]
Then we have
\\[0.2cm]
\hspace*{1.3cm}
$F(\mathbf{x}) = \mathbf{w} \cdot \mathbf{x} + b = \mathbf{w}' \cdot \mathbf{x}'$.
\\[0.2cm]
Hence, the bias has been incorporated into the weight vector at the cost of appending a $1$ at the
input vector.  As we want to use this simplification, from now on we assume that the input vectors
$\mathbf{x}^{(i)}$ have all been extended so that there last component is $1$.  Using this
assumption,  we define the
function $F$ as
\\[0.2cm]
\hspace*{1.3cm}
$F(\mathbf{x}) := \mathbf{x}^T \cdot \mathbf{w}$.
\\[0.2cm]
Now equation (\ref{eq:squared-error-1}) can be rewritten as follows:
\begin{equation}
  \label{eq:squared-error-2}
  E(\mathbf{w}) = \frac{1}{n} \cdot \sum\limits_{i=1}^n \Bigl(\bigl(\mathbf{x}^{(i)})^T \cdot \mathbf{w}  - y^{(i)}\Bigr)^2.
\end{equation}
Our aim is to rewrite the sum appearing in this equation as a scalar product of a vector with
itself.  To this end, we first define the vector $\mathbf{y}$ as follows:
\\[0.2cm]
\hspace*{1.3cm}
$\mathbf{y} := \langle y^{(1)}, \cdots, y^{(m)} \rangle^T$.
\\[0.2cm]
Note that $\mathbf{y} \in \mathbb{R}^n$ since it has a component for all of the $n$ training
examples.  Next, we define the matrix $X$ as follows:
\\[0.2cm]
\hspace*{1.3cm}
$X := \left(
  \begin{array}{c}
    \bigl(\mathbf{x}^{(1)}\bigr)^T  \\
    \vdots                         \\
    \bigl(\mathbf{x}^{(n)}\bigr)^T
  \end{array}
  \right)   
$
\\[0.2cm]
Defined this way, the row vectors of the matrix $X$ are the vectors $\mathbf{x}^{(i)}$ transposed.
Now we have the following:
\\[0.2cm]
\hspace*{1.3cm}
$X \cdot \mathbf{w} - \mathbf{y} = \left(
  \begin{array}{c}
    \bigl(\mathbf{x}^{(1)}\bigr)^T  \\
    \vdots                         \\
    \bigl(\mathbf{x}^{(n)}\bigr)^T
  \end{array}
  \right) \cdot \mathbf{w} - \mathbf{y} = \left(
  \begin{array}{c}
    \bigl(\mathbf{x}^{(1)}\bigr)^T \cdot \mathbf{w} - y_1 \\
    \vdots                         \\
    \bigl(\mathbf{x}^{(n)}\bigr)^T \cdot \mathbf{w} - y_n
  \end{array}
  \right)
$
\\[0.2cm]
Taking the square of the vector $X \cdot \mathbf{w} - \mathbf{y}$ we discover that
we can rewrite equation (\ref{eq:squared-error-2}) as follows:
\begin{equation}
  \label{eq:squared-error-3}
  E(\mathbf{w}) = \frac{1}{n} \cdot \bigl(X \cdot \mathbf{w} - \textbf{y}\bigr)^T \cdot 
                                    \bigl(X \cdot \mathbf{w} - \textbf{y}\bigr).
\end{equation}

\section{Some Useful Gradients}
In the last section, we have computed the squared error $E(\mathbf{w})$ using equation
(\ref{eq:squared-error-3}).  Our goal is to minimize the $E(\mathbf{w})$ by choosing the weight
vector $\mathbf{w}$ appropriately.  A necessary condition for $E(\mathbf{w})$ to be minimal is 
\\[0.2cm]
\hspace*{1.3cm}
$\nabla E(\mathbf{w}) = \mathbf{0}$,
\\[0.2cm]
i.e.~the gradient of $E(\mathbf{w})$ need to be zero.  In order to prepare for the computation of
$\nabla E(\mathbf{w})$, we first compute the gradient of two simpler functions.

\subsection{Computing the Gradient of $f(\mathbf{x}) = \mathbf{x}^T \cdot C \cdot \mathbf{x}$}
Suppose the function $f:\mathbb{R}^n \rightarrow \mathbb{R}$ is defined as
\\[0.2cm]
\hspace*{1.3cm}
$f(\mathbf{x}) := \mathbf{x}^T \cdot C \cdot \mathbf{x}$ \quad where $C \in \mathbb{R}^{n \times n}$.
\\[0.2cm]
If we write the matrix $C$ as $C = (c_{i,j})_{i=1,\cdots,n \atop j=1,\cdots,n}$ and the vector
$\mathbf{x}$ as $\mathbf{x} = \langle x_1, \cdots, x_n \rangle^T$,  then $f(\mathbf{x})$ can be
computed as follows:
\\[0.2cm]
\hspace*{1.3cm}
$\ds f(\mathbf{x}) = \sum\limits_{i=1}^n x_i \cdot \sum\limits_{j=1}^n c_{i,j} \cdot x_j 
                   = \sum\limits_{i=1}^n \sum\limits_{j=1}^n x_i \cdot c_{i,j} \cdot x_j
$.
\\[0.2cm]
We compute the partial derivative of $f$ with respect to $x_k$ and use the product rule:
\\[0.2cm]
\hspace*{1.3cm}
$
\begin{array}[t]{lcl}
\ds \frac{\partial f}{\partial x_k} & = &
\ds \sum\limits_{i=1}^n \sum\limits_{j=1}^n \Bigl(
    \frac{\partial x_i}{\partial x_k} \cdot c_{i,j} \cdot x_j + x_i \cdot c_{i,j} \cdot \frac{\partial x_j}{\partial x_k}
    \Bigr) \\[0.5cm]
& = &
\ds \sum\limits_{i=1}^n \sum\limits_{j=1}^n \Bigl(
    \delta_{i,k} \cdot c_{i,j} \cdot x_j + x_i \cdot c_{i,j} \cdot \delta_{j,k} \Bigr) \\[0.5cm]
& = &
\ds \sum\limits_{j=1}^n c_{k,j} \cdot x_j + \sum\limits_{i=1}^n x_i \cdot c_{i,k} \\[0.5cm]
& = &
  \bigl(C \cdot \mathbf{x}\bigr)_k + \bigl(C^T \cdot \mathbf{x}\bigr)_k
\end{array}
$
\\[0.2cm]
Hence we have shown that 
\\[0.2cm]
\hspace*{1.3cm}
$\ds \nabla f(\mathbf{x}) = (C + C^T) \cdot \mathbf{x}$.
\\[0.2cm]
If the matrix $C$ is symmetric, i.e.~if $C = C^T$, this simplifies to
\\[0.2cm]
\hspace*{1.3cm}
$\ds \nabla f(\mathbf{x}) = 2 \cdot C \cdot \mathbf{x}$.
\\[0.2cm]
If the function $g: \mathbb{R}^n \rightarrow \mathbb{R}$ is defined as 
\\[0.2cm]
\hspace*{1.3cm}
$g(\mathbf{x}) := \mathbf{b}^T \cdot A \cdot \mathbf{x}$, \quad where $\mathbf{b} \in \mathbb{R}^n$ and $A \in \mathbb{R}^{n \times n}$,
\\[0.2cm]
then a similar calculation shows that
\\[0.2cm]
\hspace*{1.3cm}
$\ds \nabla g(\mathbf{x}) = A^T \cdot \mathbf{b}$.

\exercise
Prove this equation.

\section{Deriving the Normal Equation}
Next, we derive the so called \emph{normal equation} for linear regression.  To this end, we first
expand the product in equation (\ref{eq:squared-error-3}):
\\[0.2cm]
\hspace*{1.3cm}
$
\begin{array}[t]{lcll}
 E(\mathbf{w}) & = & 
 \ds \frac{1}{n} \cdot \bigl(X \cdot \mathbf{w} - \textbf{y}\bigr)^T \cdot \bigl(X \cdot \mathbf{w} - \textbf{y}\bigr) 
 \\[0.5cm]
 & = & 
 \ds \frac{1}{n} \cdot \bigl(\mathbf{w}^T \cdot X^T - \textbf{y}^T\bigr) \cdot \bigl(X \cdot \mathbf{w} - \textbf{y}\bigr) 
 & \mbox{since $(A \cdot B)^T = B^T \cdot A^T$}
 \\[0.5cm]
 & = & 
 \ds \frac{1}{n} \cdot \bigl(\mathbf{w}^T \cdot X^T \cdot X \cdot \mathbf{w} 
                             - \textbf{y}^T \cdot X \cdot \mathbf{w} 
                             - \mathbf{w}^T \cdot X^T \cdot \mathbf{y}
                             + \mathbf{y}^T \cdot \mathbf{y}
                       \bigr)
 \\[0.5cm]
 & = & 
 \ds \frac{1}{n} \cdot \bigl(\mathbf{w}^T \cdot X^T \cdot X \cdot \mathbf{w} 
                             - 2 \cdot \textbf{y}^T \cdot X \cdot \mathbf{w} 
                             + \mathbf{y}^T \cdot \mathbf{y}
                       \bigr)
 & \mbox{since $\mathbf{w}^T \cdot X^T \cdot \mathbf{y} = \textbf{y}^T \cdot X \cdot \mathbf{w}$}
\end{array}
$
\\[0.2cm]
The fact that 
\\[0.2cm]
\hspace*{1.3cm}
$\mathbf{w}^T \cdot X^T \cdot \mathbf{y} = \textbf{y}^T \cdot X \cdot \mathbf{w}$
\\[0.2cm]
might not be immediately obvious.  This is equation is true because the result of the matrix product 
$\mathbf{w}^T \cdot X^T \cdot \mathbf{y}$ is a real number.  Now the transpose $r^T$ of a real number $r$ is the number
itself, i.e.~$r^T = r$ for all $r \in \mathbb{R}$.  Therefore, we have
\\[0.2cm]
\hspace*{1.3cm}
$\mathbf{w}^T \cdot X^T \cdot \mathbf{y} = 
\bigl(\mathbf{w}^T \cdot X^T \cdot \mathbf{y}\bigr)^T =
\mathbf{y}^T \cdot X \cdot \mathbf{w}
$.
\\[0.2cm]
Hence we have shown that
\\[0.2cm]
\hspace*{1.3cm}
\begin{equation}
  \label{eq:squared-error-4}
  E(\mathbf{w}) = \ds \frac{1}{n} \cdot \Bigl(\mathbf{w}^T \cdot \bigl(X^T \cdot X\bigr) \cdot \mathbf{w} 
                                             - 2 \cdot \textbf{y}^T \cdot X \cdot \mathbf{w} 
                                             + \mathbf{y}^T \cdot \mathbf{y}
                                        \Bigr)
\end{equation}
\\[0.2cm]
holds.  The matrix $X^T \cdot X$ used in the first term is symmetric because
\\[0.2cm]
\hspace*{1.3cm}
$\bigl(X^T \cdot X\bigr)^T = X^T \cdot \bigl(X^T\bigr)^T = X^T \cdot X$.
\\[0.2cm]
Using the results from the previous section we can now compute the gradient of $E(\mathbf{w})$ with respect to
$\mathbf{w}$.  The result is
\\[0.2cm]
\hspace*{1.3cm}
$\ds \nabla E(\mathbf{w}) = \frac{2}{n} \cdot \Bigl(X^T \cdot X \cdot \mathbf{w} - X^T \cdot \mathbf{y}\Bigr)$.
\\[0.2cm]
If the squared error $E(\mathbf{w})$ has a minimum for the weights $\mathbf{w}$, then we must have
\\[0.2cm]
\hspace*{1.3cm}
$\nabla E(\mathbf{w}) = \mathbf{0}$.
\\[0.2cm]
This leads to the equation
\\[0.2cm]
\hspace*{1.3cm}
$\ds\frac{2}{n} \cdot \Bigl(X^T \cdot X \cdot \mathbf{w} - X^T \cdot \mathbf{y}\Bigr) = \mathbf{0}$.
\\[0.2cm]
This equation can be rewritten as
\begin{equation}
  \label{eq:normal-equation}
  \bigl(X^T \cdot X\bigr) \cdot \mathbf{w} = X^T \cdot \mathbf{y}.          
\end{equation}
This equation is called the \emph{normal equation}.
Now, if the matrix $X^T \cdot X$ is invertible, then this equation can be rewritten as
\\[0.2cm]
\hspace*{1.3cm}
\colorbox{red}{\framebox{\colorbox{yellow}{\framebox{
$\mathbf{w} = \bigl(X^T \cdot X)^{-1} \cdot X^T \cdot \mathbf{y}$.
}}}}
\\[0.2cm]
In this case, we define 
\\[0.2cm]
\hspace*{1.3cm}
$X^+ := \bigl(X^T \cdot X)^{-1} \cdot X^T$.
\\[0.2cm]
The expression $X^+$ is called the 
\href{https://en.wikipedia.org/wiki/Moore–Penrose_pseudoinverse}{\emph{Moore-Penrose pseudoinverse}} of the matrix $X$.
We can understand in what sense $X^+$ is an inverse of $X$ by noting that if $X$ itself were invertible, we could solve
the equation
\\[0.2cm]
\hspace*{1.3cm}
$X \cdot \mathbf{w} = \mathbf{y}$
\\[0.2cm]
by defining
\\[0.2cm]
\hspace*{1.3cm}
$\mathbf{w} := X^{-1} \cdot \mathbf{y}$.
\\[0.2cm]
In that case, the squared error would be zero.
In general, $X$ will not be invertible for the simple reason that the matrix $X$ is an $n \times m$ matrix and hence is
not even a square matrix if $m \not= n$.  In practical applications of linear regression the number $n$ of training
examples is bigger than the dimension $m$ of the vectors $\mathbf{x}_i$ that make up the training examples.  Hence there
is no hope of $X$ being invertible.  The next best thing is then to minimize the squared error.  To do this, we take the
previous equation for the weight vector $\mathbf{w}$ and replace $X^{-1}$ with the pseudoinverse of $X$.  This gives the
equation 
\\[0.2cm]
\hspace*{1.3cm}
$\mathbf{w} = \bigl(X^T \cdot X)^{-1} \cdot X^T \cdot \mathbf{y}$
\\[0.2cm]
derived previously.

At this point you might ask whether the matrix $X^T \cdot X$ is always invertible.  The short answer is yes, provided the
regression problem is well posed.  By this I mean that there are much more training examples than there are different
weights.  After all, if you have only 10 training examples but want to determine a 100 different weights, then it is
obvious that the problem is not well posed.  Now if you have enough training examples you have to be sure that the
training examples are independent of each other.  For example, lets say you have three training examples 
$\langle\mathbf{x}_1, y^{(1)}\rangle$, $\langle\mathbf{x}_2, y^{(2)}\rangle$, and $\langle\mathbf{x}_3, y^{(3)}\rangle$ and
you decide that you define a fourth training example $\langle\mathbf{x_4}, y^{(4)}\rangle$ as
\\[0.2cm]
\hspace*{1.3cm}
$\mathbf{x}_4 := \mathbf{x}_1 + \mathbf{x}_2 + \mathbf{x}_3$ \quad and \quad 
$y^{(4)} := y^{(1)} + y^{(2)} + y^{(3)}$,
\\[0.2cm]
then you have not generated any new information and the fourth training example will not help in making the matrix 
$X^T \cdot X$ invertible.

%%% Local Variables:
%%% mode: latex
%%% TeX-master: "artificial-intelligence"
%%% End:
